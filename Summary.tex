\documentclass[11pt, twocolumn]{article}
\pagestyle{empty}
\usepackage{amsmath}
\usepackage{amssymb}
\usepackage{graphicx}
\usepackage{amsthm}
\usepackage[margin=1cm]{geometry}
\usepackage{setspace}
\setstretch{0.7}
\newenvironment{compactitem}
{\begin{itemize}
  \setlength{\itemsep}{1px}
  \setlength{\parskip}{0pt}
  \setlength{\parsep}{0pt}}
{\end{itemize}}

\begin{document}
\section{Natural Deduction}
\begin{compactitem}
\item Conjunction introduction: $F_1, F_2 \vdash F_1 \wedge F_2$
\item Conjunction elimination: $F_1 \wedge F_2 \vdash F_1$
\item Conjunction elimination: $F_1 \wedge F_2 \vdash F_2$
\item Disjunction introduction: $F_1 \vdash F_1\vee F_2$
\item Disjunction introduction: $F_2 \vdash F_1\vee F_2$
\item Disjunction elimination: $F_1 \vee F_2, F_1\Rightarrow F_3, F_2\Rightarrow F_3 \vdash F_3$
\item Negation introduction: $F_2 \Rightarrow (F_1 \wedge \neg F_1) \vdash \neg F_2$
\item Double negation elimination: $\neg \neg F_1 \vdash F_1$
\item Implication introduction: $(F_1 \vdash F_2)\vdash F_1\Rightarrow F_2$
\item Double negation introduction: $F_1 \vdash \neg\neg F_1$
\item Implication elimination (Modus Ponens): $F_1, F_1 \Rightarrow F_2 \vdash F_2$
\item Modus Tollens: $F_1\Rightarrow F_2, \neg F_2 \vdash \neg F_1$
\item Universal quantifier elimination: $\forall X F \vdash F[t/X]$
\item Universal quantifier introduction: $F[Y/X] \vdash \forall X F$
\item Existential quantifier introduction: $F[t/X] \vdash \exists X F$
\item Existential quantifier elimination: $\exists X F_1, (F_1[Y/X] \vdash F_2) \vdash F_2$
\end{compactitem}
\section{Boolean Algebra}
\subsection{Axiom}
\begin{compactitem}
\item Identity of $\times$: $x \times 1 = x$
\item Identity of $+$: $x +0 = x$
\item Complementation of $\times$: $x \times \bar{x} =0$
\item Complementation of $+$: $x+ \bar{x} = 1$
\item Associativity of $\times$: $x\times(y\times z) = (x \times y)\times z=x\times y \times z$
\item Associativity of $+$: $x+(y+z)=(x+y)+z=x+y+z$
\item Commutativity of $\times$: $x\times y = y\times x$
\item Commutativity of $+$: $x+y=y+x$
\item Distributivity of $\times$ over $+$: $x\times(y+z)= (x\times y) +(x\times z)$
\item Distributivity of $+$ over $\times$: $x+(y\times z)= (x+y) \times (x+z)$
\end{compactitem}
\subsection{Theorems}
\begin{compactitem}
\item Idempotence of $\times$: $x\times x = x$
\item Idempotence of $+$: $x+x=x$
\item Annihilator of $\times $: $x\times 0=0$
\item Annihilator of $+$: $x+1=1$
\item Absorption of $\times$: $x\times (x+y) = x$
\item Absorption of $+$: $x+(x\times y) =x$
\item Unicity of complement: If $x\times y = 0$ and $x+y=1$ then $y=\bar{x}$
\end{compactitem}
\section{Sets}
\begin{compactitem}
\item Empty set. There exist a set with no elements $\exists X (\forall Y(\neg(Y\in X)))$
\item Extensionality: Two sets are equal if and only if they have the same elements $\forall X\forall Y ((\forall Z(Z \in X \Leftrightarrow Z\in Y)) \Leftrightarrow X=Y)$
\item Pairing: There exists a set $Z$ that contains $X$ and $Y$. $\forall X \forall Y \exists Z \forall T ((T=X\vee T=Y)\Leftrightarrow T\in Z)$
\item Unions: $\forall S \exists T \forall Y ((Y\in T)\Leftrightarrow \exists Z((Z\in S)\wedge (Y\in Z)))$
\item Power Sets: $\forall S \exists T \forall X (( X\in T) \Leftrightarrow (X\subset S))$
\item Regularity: $\forall X (X \neq \varnothing \Rightarrow (\exists Y (Y \in X \wedge \forall Z (Z\in X \Rightarrow \neg (Z\in Y)))))$
\item Infinity: $\exists X(\varnothing \in X \wedge (\forall Y (Y\in X \Rightarrow Y\cup \{ Y \} \in X )))$
\item Separation: $ \forall X \exists Y \forall Z (Z\in Y \Leftrightarrow (Z\in X \wedge P(Z)))$
\end{compactitem}
\section{Functions}
\begin{compactitem}
\item Uniqueness: $\exists!x\in T p(x) \equiv \exists x \in T (p(x) \wedge \forall y \in T (p(y) \Rightarrow x=y))$ 
\item Definition of function: $f: S \rightarrow T \Leftrightarrow \forall x \in S \;\exists y \in T (x\;f\;y \wedge (\forall z \in T (x\;f\;z \Rightarrow y=z)))$ or $\forall x\in S \;\exists!y \in T (x\;f\;y)$
\item Notation: $f(x)=y \Leftrightarrow <x,y>\in f$
\item $Dom(f) = S$
\item $x$ is a pre-image of $y$ if $f(x)=y$ while the inverse image of $y$ is the set of all its pre-images.
\item The restriction of $f$ to $U$ is the set $\{<x,y>\in U\times T\;|\;f(x)=y\}$
\item Injection: $\forall y \in T \;\forall x_1 \in S \;\forall x_2 \in S \;((f(x_1)=y\wedge f(x_2)=y)\Rightarrow x_1=x_2)$ 
\item Surjection: $\forall y \in T \; \exists x \in S \; (f(x)=y)$, $Im(f)=T$
\item Bijection = Injection + Surjection
\item $f$ is bijective iff $f^{-1}$ is a function
\item Notation: $g(f(x)) = (g\circ f)(x)$. If $f: S \rightarrow T$ and $g: T \rightarrow U$ are both functions, then $g\circ f$ is a function.
\item Identity function, $\mathcal{I}_A$: $\forall x \in A (\mathcal{I}_A(x)=x)$ 
\item If $f: A \rightarrow A$ is injective, then $f^{-1} \circ f = \mathcal{I}_A$
\item A $n$-ary operation on a set $A$ is a function $f: \prod^n_1A\rightarrow A$
\end{compactitem}
\section{Number System}
\begin{compactitem}
\item Peano's Axioms: The set of natural numbers is the smallest set such that:
\begin{compactitem}
\item $\exists 0 \; (0 \in \mathbb{N})$
\item There exist a function $s$ on $\mathbb{N}$. $s(n) = n'$, the successor of $n$.
\item $\forall n \in \mathbb{N} \;(n' \neq 0)$
\item $n \in \mathbb{N}\; \forall m \in \mathbb{N} \; (n' = m' \Rightarrow n=m)$
\item $\forall K \subset \mathbb{N} \; \forall n\in \mathbb{N} \;((0\in K \wedge (n \in K \Rightarrow n' \in K))\Rightarrow K = \mathbb{N} )$  
\end{compactitem}
\item Ordering Lemma: Let $m\in \mathbb{N}_c$. Then $n\leq m \vee m\leq n$.
\item Addition is the smallest binary operation on $\mathbb{N}$ such that:
\begin{compactitem}
\item $\forall n \in \mathbb{N} \;(n+0=n)$
\item $ \forall n\in \mathbb{N} \; \forall m\in \mathbb{N} (n+m' = (n+m)')$
\end{compactitem}
\item Multiplication is the smallest binary operation on $\mathbb{N}$ such that:
\begin{compactitem}
\item $\forall n \in \mathbb{N} \; (n\times 0 = 0)$
\item $\forall n \in \mathbb{N} \forall m \in \mathbb{N} (n\times m' = (n\times m)+n)$
\end{compactitem}
\item Integers
\begin{compactitem}
\item Let $\approx$ be a relation on $\mathbb{N} \times \mathbb{N}$ such that $\forall n_1 \in \mathbb{N} \; \forall n_2 \in \mathbb{N} \; \forall m_1 \in \mathbb{N} \; \forall m_2 \in \mathbb{N} \;(<n_1, n_2> \approx <m_1, m_2> \Leftrightarrow n_2+m_1=m_2+n_1)$. The set of integers is given by $\mathbb{Z} = (\mathbb{N} \times \mathbb{N})/\approx$
\item $a+b = <a_1+b_1, a_2+b_2>$
\item $a-b = <a_1+b_2, a_2+b_1>$
\item $a\times b = <a_1\times b_2+a_2\times b_1, a_1\times b_1+a_2\times b_2 >$
\end{compactitem}
\item Rationals
\begin{compactitem}
\item Let $\approx$ be a relation on $\mathbb{N} \times \mathbb{N}\backslash\{0\}$ such that $\forall n_1 \in \mathbb{N} \; \forall n_2 \in \mathbb{N}\backslash \{0\} \; \forall m_1 \in \mathbb{N} \; \forall m_2 \in \mathbb{N}\backslash\{0\} \;(<n_1, n_2> \approx <m_1, m_2> \Leftrightarrow n_2\times m_1=m_2\times n_1)$. The set of rationals is given by $\mathbb{Z} = (\mathbb{N} \times \mathbb{N}\backslash \{0\})/\approx$
\item $a\times b = <a_1\times b_1, a_2\times b_2>$
\item $a+b = <(a_1\times b_2+b_1\times a_2, a_2\times b_2>$
\item $a-b = <(a_1\times b_2-b_1\times a_2, a_2\times b_2>$
\item $a/b = <a_1\times b_2, a_2\times b_1>$
\end{compactitem}
\end{compactitem}
\section{Number Theory}
\begin{compactitem}
\item Let $n\in \mathbb{N}$ and $m\in \mathbb{N}^+$. $m | n \Leftrightarrow \exists q \in \mathbb{N} \; (n=m\times q)$
\item Division algorithm: Let $n\in \mathbb{N}$, $m\in \mathbb{N}^+$, $\exists !q\in \mathbb{N} \; \exists !r\in \mathbb{N} \;(n=q\times m+r \wedge r<m)$
\item $n$ and $m$ are relatively prime ($n\perp m$) iff $\forall c\in \mathbb{N}^+ (((c|n)\wedge (c|m))\Rightarrow c=1)$
\item $p$ is prime if $p>1 \wedge (\forall n \in \mathbb{N}^+\;(n|p\Rightarrow (n=p\vee n=1)))$
\item The gcd of $n$ and $m$ is defined such that $(\text{gcd}(n, m) |n) \wedge (\text{gcd}(n,m) |m)\wedge (\forall q\in \mathbb{N}^+ \;(((q|n)\wedge (q|m))\Rightarrow q\leq \text{gcd}(n,m)))$
\item The lcm of $n$ and $m$ is defined such that $(n|\text{lcm}(n, m)) \wedge (m|\text{lcm}(n,m))\wedge (\forall q\in \mathbb{N}^+ \;(((n|q)\wedge (m|q))\Rightarrow \text{lcm}(n,m)\leq q))$
\item Bezout's Identity: Let $n \in \mathbb{N}^+$, $m \in \mathbb{N}^+$. $\exists a \in \mathbb{Z} \; \exists b\in \mathbb{Z} \; (n\times a + m\times b = \text{gcd}(m,n))$
\item Euclid's Lemma: Let $n \in \mathbb{N}^+$, $m \in \mathbb{N}^+$. Let $p\in \mathbb{N}^+$. $(\text{prime}(p)\wedge (p|n\times m))\Rightarrow (p|n \vee p|m)$
\item Factorisation: $n = \prod_{i\in I} p_i$
\item Fundamental theorem of arithmetic: Every positive natural number has a unique factorisation.
\item $n \equiv m \pmod{c} \Leftrightarrow (m<n\wedge c|n-m)\vee (n<m\wedge c|m-n)\vee (n=m)$
\item If $n_1\equiv m_1 \pmod{c} \wedge n_2 \equiv m_2 \pmod{c}$, then $n_1+n_2 \equiv m_1+m_2 \pmod{c}$
\item If $n_1\equiv m_1 \pmod{c} \wedge n_2 \equiv m_2 \pmod{c}$, then $n_1\times n_2 \equiv m_1\times m_2 \pmod{c}$
\item Fermat's Little Theorem: If $p$ is prime then $a^p\equiv a\pmod{p}$
\end{compactitem}
\section{Cardinality}
\begin{compactitem}
\item A set is finite iff it is the empty set or there is a bijection from the set to some $\mathbb{Z}_n$
\item Cardinality of a finite set is 0 if the set is empty set or $n$ if there is a bijection from the set to some $\mathbb{Z}_n$.
\item $\aleph_0 \equiv |\mathbb{Z}^+|$
\item A set is countably infinite iff $|S| = \aleph_0$. Otherwise it is uncountable. A set is countably infinite iff its elements can be arranged without duplication and omission in an infinite list.
\item Given $n$ countably infinite sets, the cartesian product of these sets are also countably infinite
\item The union of countably many countable sets is countable
\item The set of real numbers is uncountable
\item $|A|<|\mathcal{P}(A)|$
\item $|\mathbb{R}|=|\mathcal{P}(\mathbb{Z})|$
\end{compactitem}
\section{Counting}
\begin{compactitem}
\item Product Rule: If $A_1, A_2, \cdots A_n$ are finite sets, then $|\prod_{i=1}^n A_i| = \prod_{i=1}^n|A_i|$.
\item Sum Rule: If $A_1, A_2, \cdots A_n$ are disjoint finite sets, then $|\bigcup_{i=1}^n A_i| = \sum_{i=1}^n|A_i|$.
\item Difference Rule: if $A$ is a finite set and $B$ is a subset of $A$, then $|A-B| = |A|-|B|$.
\item Principle of Inclusion and Exclusion: $|\bigcup_{i=1}^n A_i| = \sum_{1\leq i\leq n}|A_i|-\sum_{1\leq i\leq j\leq n}|A_i\cap A_j|+\sum_{1\leq i\leq j \leq k \leq n}|A_i\cap A_j\cap A_k|-\cdots +(-1)^{n+1}|\bigcap_{i=1}^n A_i|$
\item The number of permutations of a set of $n$ distinct elements is $n!$
\item The number of ordered selection of $r$ objects from $n$ objects is $P(n, r)=\frac{n!}{(n-r)!}$
\item The number of unordered selection of $r$ objects from $n$ objects is $C(n,r) = \frac{n!}{(r!(n-r)!}$
\item The number of permutations of $n$ objects where there are $n_i$ indistinguishable objects of type $i$ for $i=1, 2, \cdots k$ and $\sum_{i=1}^{k} n_1 = n$, is $\frac{n!}{n_1!n_2!\cdots n_k!}$ 
\item The number of $r$-combinations with repetition allowed that can be selected from $n$ items is $C(n+r-1, r)$
\item Binomial theorem: $(x+y)^n = \sum^n_{r=0} \binom{n}{r}x^{n-r}y^r$
\item Pascal's identity: $\binom{n+1}{k} = \binom{n}{k-1} + \binom{n}{k}$
\item Generalised Pigeonhole Principle: If $N$ objects are placed into $k$ boxes, there is at least 1 box which contain at least $\lceil \frac{N}{k} \rceil$ objects.
\end{compactitem}
\section{Graphs}
\begin{compactitem}
\item A pseudograph $G = (V(G), E(G), f_G)$ consists of 
\begin{compactitem}
\item A non-empty vertex set of vertices
\item An edge set of edges
\item An incidence function $f_G: E(G) \rightarrow \{\{u, v\}|u, v\in V(G)\}$
\end{compactitem}
\item An edge $e$ is called a loop if $f_G(e) = \{u\}$ for some $u\in V(G)$
\item If $f_G(e_1) = f_G(e_2)$, then the two edges are called multiple or parallel edges
\item A simple graph has no loops or parallel edges, a multigraph has no loops
\item An edge $e$ is incident with vertices $u$ and $v$ if $f_G(e) = \{ u, v \}$. The two vertices are adjacent. Two edges are called adjacent if they have are incident with a common vertex.
\item A directed multigraph $G = (V(G), E(G), f_G)$ consists of 
\begin{compactitem}
\item A non-empty vertex set of vertices
\item An edge set of directed edges
\item An incidence function $f_G: E(G) \rightarrow \{(u, v)|u, v\in V(G)\}$
\end{compactitem}
\item A complete graph is a simple graph where every two distinct vertices are adjacent
\item A complete bipartite graph on $(m, n)$ vertices, denoted by $K_{m,n}$ is a simple graph with $V(K_{m, n}) = \{u_1, u_2, \cdots u_m\} \cup \{v_1, \cdots v_n\}$ and $E(K_{m, n}) = \{\{u_i, v_j\}|i=1, \cdots, m; j = 1, \cdots n\}$
\item A graph $H$ is a subgraph of $G$ if $V(H)\subseteq V(G)$, $E(H) \subseteq E(G)$ and $\forall e\in E(H)\; (f_H(e) = f_G(e))$
\item The vertex degree $d_G(v)$ of vertex $v$ is the number of edges incident with $v$
\item Handshake Theorem: Let $G$ be an undirected graph. Then $\sum_{v\in V(G)} d_G(v) = 2|E(G)|$
\item A walk of length $n$ in $G$ is defined as an alternating sequence of vertices and edges of $G$ $v_0e_1v_1e_2\cdots e_n v_n$ where $e_i$ connects its endpoints $v_{i-1}$ and $v_{i}$. A trail is a walk of distinct edges while a path is a trail with distinct vertices.
\item A graph is connected if there is a walk between any two vertices in the graph.
\item An undirected graph $H$ is called a connected component of $G$ if $H$ is a subgraph of $G$, $H$ is connected and no connected subgraph of $G$ has $H$ as its proper subgraph.
\item An Euler trail is a trial traversing all edges in a graph. A closed walk is a walk that starts and ends at the same vertex. A tour is a closed walk that traverse all edges in a graph at least once. An Euler tour is a tour that traverse each edge exactly once.
\item A graph has an Euler tour if and only if it has no vertices of odd degree. A graph has an Euler trail but not an Euler tour if and only if it has exactly two vertices of odd degree.
\item A Hamiltonian path is a path containing all vertices in $G$. A cycle is a closed trail whose origin and internal vertices are distinct. A Hamiltonian cycle is a cycle which contains all vertices of G.
\item Let $G$ be an undirected graph and the vertices are ordered as $V(G) = \{v_1, v_2, \cdots v_n\}$. The adjacency matrix of $G$ is the $n\times n$ matrix $A(G) = [a_{ij}]$ such that $a_{ij} = |\{e\in E(G) | f_G(e) = \{v_i, v_j\}\}|$
\end{compactitem}
\section{Trees}
\begin{compactitem}
\item A tree is an acyclic connected undirected graph. A forest is an acyclic undirected graph.
\item There is a unique path between any two distinct vertices in a graph iff the graph is a tree
\item If $G$ is a tree iff $|E(G)| = |V(G)|-1$
\item A graph $H$ is a subgraph of graph $G$ if $V(H)\subseteq v(G)$, $E(H) \subseteq E(G)$ and $\forall E(H)\; f_H(e) = f_G(e)$
\item A spanning subgraph of a graph is a subgraph of a graph which contains all vertices in the graph. A spanning tree is a spanning subgraph which is a tree
\item A graph is connected if and only if there is a spanning tree
\item A weighted graph is a graph where all edges has an associated real valued weight
\item A minimum spanning tree of a connected weighted graph is the spanning tree that has the smallest possible sum of weights in its edges
\end{compactitem}
\end{document}  